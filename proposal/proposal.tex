\documentclass{article} % For LaTeX2e
\usepackage{nips13submit_e,times}

\usepackage{hyperref}
\usepackage{url}
\usepackage{graphicx}
\usepackage{chronology}
\usepackage[T1]{fontenc}
\usepackage[utf8]{inputenc}
\usepackage{charter}
\usepackage{environ}
\usepackage{tikz}
\usetikzlibrary{calc,matrix}
%\documentstyle[nips13submit_09,times,art10]{article} % For LaTeX 2.09


\title{Applying Reinforcement Learning to Wordle}


\author{
Julius Breindl \\
CSE, University at Buffalo\\
\texttt{jbreindl@buffalo.edu} \\
\And
Andrew Sipper \\
CSE, University at Buffalo\\
\texttt{asipper@buffalo.edu} \\
\And
Vivek Shankar \\
CSE, University at Buffalo\\
\texttt{shankar9@buffalo.edu} \\
}


\makeatletter
\let\matamp=&
\catcode`\&=13
\makeatletter
\def&{\iftikz@is@matrix
  \pgfmatrixnextcell
  \else
  \matamp
  \fi}
\makeatother

\newcounter{lines}
\def\endlr{\stepcounter{lines}\\}

\newcounter{vtml}
\setcounter{vtml}{0}

\newif\ifvtimelinetitle
\newif\ifvtimebottomline
\tikzset{description/.style={
  column 2/.append style={#1}
 },
 timeline color/.store in=\vtmlcolor,
 timeline color=red!80!black,
 timeline color st/.style={fill=\vtmlcolor,draw=\vtmlcolor},
 use timeline header/.is if=vtimelinetitle,
 use timeline header=false,
 add bottom line/.is if=vtimebottomline,
 add bottom line=false,
 timeline title/.store in=\vtimelinetitle,
 timeline title={},
 line offset/.store in=\lineoffset,
 line offset=4pt,
}

\NewEnviron{vtimeline}[1][]{%
\setcounter{lines}{1}%
\stepcounter{vtml}%
\begin{tikzpicture}[column 1/.style={anchor=east},
 column 2/.style={anchor=west},
 text depth=0pt,text height=1ex,
 row sep=1ex,
 column sep=1em,
 #1
]
\matrix(vtimeline\thevtml)[matrix of nodes]{\BODY};
\pgfmathtruncatemacro\endmtx{\thelines-1}
\path[timeline color st] 
($(vtimeline\thevtml-1-1.north east)!0.5!(vtimeline\thevtml-1-2.north west)$)--
($(vtimeline\thevtml-\endmtx-1.south east)!0.5!(vtimeline\thevtml-\endmtx-2.south west)$);
\foreach \x in {1,...,\endmtx}{
 \node[circle,timeline color st, inner sep=0.15pt, draw=white, thick] 
 (vtimeline\thevtml-c-\x) at 
 ($(vtimeline\thevtml-\x-1.east)!0.5!(vtimeline\thevtml-\x-2.west)$){};
 \draw[timeline color st](vtimeline\thevtml-c-\x.west)--++(-3pt,0);
 }
 \ifvtimelinetitle%
  \draw[timeline color st]([yshift=\lineoffset]vtimeline\thevtml.north west)--
  ([yshift=\lineoffset]vtimeline\thevtml.north east);
  \node[anchor=west,yshift=16pt,font=\large]
   at (vtimeline\thevtml-1-1.north west) 
   {\textsc{Timeline \thevtml}: \textit{\vtimelinetitle}};
 \else%
  \relax%
 \fi%
 \ifvtimebottomline%
   \draw[timeline color st]([yshift=-\lineoffset]vtimeline\thevtml.south west)--
  ([yshift=-\lineoffset]vtimeline\thevtml.south east);
 \else%
   \relax%
 \fi%
\end{tikzpicture}
}

\newcommand{\fix}{\marginpar{FIX}}
\newcommand{\new}{\marginpar{NEW}}

\nipsfinalcopy % Uncomment for camera-ready version


\begin{document}
\maketitle

\section{Introduction}
Wordle, created by Josh Wardle, is a game in which the player has 6 guesses to attempt to
guess a random 5 letter word. Every guess, each letter either returns as green, yellow, or
gray. If its green, the letter is where it belongs in the word. If it's yellow, the letter
is in the word, but in a different position. If its gray, the word doesn't contain the
letter~\cite{wordlewiki}. Because it already has a built in reward structure our team thought
it would be an interesting game to try to solve.

%\bigskip

Applying reinforcement learning techniques to solve games is something that the team at
Deep Mind has done multiple times~\cite{dota}~\cite{atari}~\cite{alphago}. Reinforcement learning has also been applied
to solve Scrabble~\cite{scrabble} --- a game fundamentally similar to Wordle.
There are existing programatic solutions to Wordle, the most prominent of which being the 3
blue 1 brown solution that is based on entropy minimization~\cite{solver1}.
There is also a Reinforcement Learning version that exists that uses DQN and
A2C~\cite{solver2}. This reinforcement learning solver was trained by continually increasing
the vocabulary that the bot ``knew'' until it had an entire dictionary in it.

%\bigskip

We intend to develop an OpenAI gym environment to train the agent in, and then create
multiple agents to compare performance of different algorithms on this problem. If time
allows, we will compare not only our own implementations, but some of the solutions using
traditional NLP methods.

\section{Project timeline}
\begin{vtimeline}[description={text width=8cm},
 row sep=4ex,]
1 April & Literature Review and algorithm selection\endlr
3 April & Proposal Due\endlr
5 April & Environment Completed\endlr
15 April & Algorithms Completed\endlr
17 April & Checkpoint due: benchmarks completed\endlr
1 May & Improvements completed\endlr
6 May & Demo days results due\endlr
\end{vtimeline}
\addcontentsline{toc}{sen=ction}{\refname}
\section*{References}
\bibliographystyle{plain}
\bibliography{refs}
\end{document}
